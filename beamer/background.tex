\begin{frame}
  \frametitle{Background}
  In this opening section, I will talk briefly about
  \begin{itemize}
    \item The History of Fortran
    \item Fortran 77 vs. Fortran 90
    \item Example programs
  \end{itemize}
\end{frame}

\begin{frame}
  \frametitle{History of Fortran}
  \begin{itemize}
    \item \textbf{For}mula \textbf{Tran}slation
    \item Around since 1952
    \item Procedural Programming Paradigm
    \item Versions usually denoted by year of publication, eg. 66, 77, 90, 95, 2003
    \item This course deals with versions 77 to 95.
  \end{itemize}
\end{frame}

\begin{frame}[fragile]
  \frametitle{Example Fortran77 Code}
  \begin{lstlisting}
       PROGRAM HELLO
       IMPLICIT NONE
       INTEGER i
C      This is a comment
       DO 100 i = 1, 10
       PRINT *, "I am in iteration ", i,
      & " of the loop."
 100   CONTINUE
       END PROGRAM
  \end{lstlisting}
\end{frame}

\begin{frame}
  \frametitle{Fortran 77}
  Fixed Form:
  \begin{itemize}
       \item The position in the line of a character has meaning.
       \item Anything in first position (other than digit): 
             Line is Comment.
       \item A number in postions 1-5: 
             Label that can be referenced somewhere else in the code
       \item Anything in position 6:
             Continuation line, this code continues from last line
       \item Code between position 7 and 72 (inclusive)
  \end{itemize}
\end{frame}

\begin{frame}[fragile]
  \frametitle{Example Fortran90 Code}
  \begin{lstlisting}
program hello
    implicit none
    integer :: i    ! This is a comment
    do 100 i = 1, 10
        print *, "I am in iteration ", i, & 
                 " of the loop."
    end do
end program hello
  \end{lstlisting}
\end{frame}

\begin{frame}
  \frametitle{Fortran 90}
  Free Form: Position in the line no longer relevant.
  Some added features:
  \begin{itemize}
    \item DO - END DO
    \item Modules
    \item Comments now denoted with Exclamation Mark
    \item But still mostly backwards compatible.
  \end{itemize}
\end{frame}

\begin{frame}
  \frametitle{Fortran 95}
  Fortran 95 removed items from the syntax that have been depreciated.
  Examples:
  \begin{itemize}
    \item DO loops with floating point iterators
    \item Calculating Jump destinations
  \end{itemize}
  But also new features were introduced, for example:
  \begin{itemize}
    \item Implicit Loops with FORALL and WHERE
    \item PURE, ELEMENTAL, and RECURSIVE procedures
  \end{itemize}
\end{frame}

\begin{frame}
  \frametitle{Newer Fortran Versions}
  \begin{itemize}
    \item Fortran 2003
    \begin{itemize}
      \item Some Object Oriented Features
      \item Input/Output Streams
    \end{itemize}
    \item Fortran 2008
    \begin{itemize}
      \item Easier parallelisation with DO CONCURRENT
      \item Submodules
    \end{itemize}
    \item Fortran 2015
    \begin{itemize}
      \item ???
    \end{itemize}
  \end{itemize}
\end{frame}


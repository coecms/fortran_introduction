\begin{frame}[fragile]
  \frametitle{Depreciated Features}

  This section explains a few features that are depreciated.

  You usually wouldn't want to use any of these features yourself, 
  however you might come across them when looking at old code.

\end{frame}

\begin{frame}[fragile]
  \frametitle{CONTINUE}

  The \texttt{CONTINUE} statement does nothing at all.

  It was usually used in conjunction with labels to remove any ambiguity
  as to whether the line with the label was to be executed or not.

\end{frame}

\begin{frame}[fragile]
  \frametitle{COMMON}

  Common blocks were used to share variables between different procedures without
  having to pass them through as arguments.

  It was important that all procedures that share these variables use the exact
  same variable names, and the same \texttt{COMMON} statement.

  Use \texttt{MODULES} instead.

\end{frame}

\begin{frame}[fragile]
  \frametitle{EQUIVALENCE}

  \texttt{EQUIVALENCE} means that different variables should literally
  share the same memory location.

  This was originally necessary to reuse memory on computers that had kilobytes 
  at best. 

  Modern computers have more than enough memory, so this need is no longer there.

  Some people claim that they've found other uses for \texttt{EQUIVALENCE} but you
  really shouldn't bother with that.

\end{frame}

\begin{frame}[fragile]
  \frametitle{DATA}

  \texttt{DATA} was used to initialise variables, that is to give them their initial values.

  Just assign the values at the beginning of your execution block.

\end{frame}

\begin{frame}[fragile]
  \frametitle{GOTO}

  \texttt{GOTO} is used with a label. 
  Execution of the program jumps directly to the line with the label.

  \begin{lstlisting}
print *, "This line is executed."
goto 1001
print *, "This line is not."
1001 CONTINUE
print *, "This line is executed again."
  \end{lstlisting}

  \texttt{GOTO} can jump both forwards and backwards. 
  It interrupts the code flow and is therefore despised by many programmers.

  Using loops and \texttt{IF} blocks can often achieve the same thing in a more
  elegant way.

  That said: "Real Programmers Aren't Afraid of GOTO."

\end{frame}

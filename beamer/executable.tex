\begin{frame}
  \frametitle{Writing Source Code}
  Source Code is written with a Text Editor.

  Not to be confused with Word Processors. 
  Basically: If you can easily change the Font, you can't use it to write code.

  Examples for good text editors:

  \begin{itemize}
    \item Windows: Notepad++
    \item MacOS: TextWrangler
    \item Linux GUI: gedit
    \item Linux console: vim, emacs, nano
  \end{itemize}
\end{frame}

\begin{frame}[fragile]
  \frametitle{Compiling Simple}
  Let's consider this very simple "Hello World" program:
  \begin{lstlisting}
program hello
    implicit none
    print *, "Hello World"
end program hello
  \end{lstlisting}
  Such a program can be compiled in a single step:
  \begin{lstlisting}[language=bash,numbers=none]
$ gfortran -o hello hello.f90
  \end{lstlisting}
\end{frame}

\begin{frame}[fragile]
  \frametitle{Compiling Multiple Files}
  If your code spans several files, you have to independently compile each into
  object files. Often you need to do that in a certain order.
  Then you link the object files together into the executable.

  \begin{lstlisting}[language=bash,numbers=none]
$ gfortran -c -o mod_hello.o mod_hello.f90
$ gfortran -c -o hello.o hello.f90
$ gfortran -o hello hello.o mod_hello.o
  \end{lstlisting}

  Note the \texttt{-c} flag in the first two lines.
\end{frame}

\begin{frame}[fragile]
  \frametitle{Compiler Warnings}
  Sometimes your code is syntactically correct, but doesn't do what you think it does.
  There are some warning signs that the compiler can pick up, such as 
  Variables that are declared but never used, or blocks of code that are never reached.

  It is good programming practise to always enable all warnings, and then eliminate the causes for these warnings.

  \begin{itemize}
    \item \textbf{Intel Fortran:} use option \texttt{-warn all}
    \item \textbf{GNU Fortran:} use option \texttt{-Wall}
  \end{itemize}
\end{frame}

\begin{frame}[fragile]
  \frametitle{Compiler Debugging Settings}
  Some compiler options reduce performance, but provide useful feedback during
  programming and debugging.

  \begin{table}
  \begin{tabular}{lcc}
    & \textbf{Intel} & \textbf{GNU} \cr
    Traceback & \texttt{-traceback} & \texttt{-fbacktrace} \cr
    Checking & \texttt{-check all} & \texttt{-fbounds-check}
  \end{tabular}
  \end{table}
  Once you're done debugging, compile it without these options.
\end{frame}



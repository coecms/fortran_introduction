\begin{frame}[fragile]
  \frametitle{Output and Input}

  The next Section is about Output and Input of Data.

\end{frame}

\begin{frame}[fragile]
  \frametitle{PRINT}

  The first way to output data has been featuring prominently
  on most slides:

  \begin{lstlisting}[numbers=none]
print *, <data>
  \end{lstlisting}

  This statement dumps the data to the standard output.
  The formatting isn't nice, but it's really easy.

\end{frame}

\begin{frame}[fragile]
  \frametitle{WRITE}

  To get more control over the output, the \texttt{WRITE}
  statement is used.

  \begin{lstlisting}[numbers=none]
write(*, *) <data>
  \end{lstlisting}

  This line is actually equivalent to the \texttt{print} statement.

  But note the two asterisks:

  The first one declares the destination of the write statement (* is standard output),
  the second one the format (* is standard format).

\end{frame}

\begin{frame}[fragile]
  \frametitle{Format}

  The format can be given in three ways:
  
  Directly as a string:
  \begin{lstlisting}
write(*, '(A6, I4)') "Year: ", 2015
  \end{lstlisting}

  Placed in a string variable:
  \begin{lstlisting}
fmt = "(A6, I4)"
write(*, fmt) "Year: ", 2015
  \end{lstlisting}

  Or by giving the label of a \texttt{FORMAT} statement:
  \begin{lstlisting}
write(*, 1001) "Year: ", 2015
1001 format(A6, I4)
  \end{lstlisting}

\end{frame}

\begin{frame}[fragile]
  \frametitle{Format Examples}

  \begin{table}
  \begin{tabular}{| l | l |}
    \hline
    \textbf{ID} & \textbf{Explanation} \cr
    \hline
    \texttt{In.w} & Integer, n is total length, w is min length \cr
                  & padded with 0 \cr
    \hline
    \texttt{Fn.w} & Float, n is total length, w is number of \cr
                  & digits after the comma \cr
    \hline
    \texttt{En.w} & Float in notation 2.41E4, n is total length, \cr
                  & w is number of digits after the comma \cr
    \hline
    \texttt{An}   & String, n is length \cr
    \hline
  \end{tabular}
  \end{table}

  There are lots more, google "Fortran Format" for complete list.

\end{frame}

\begin{frame}[fragile]
  \frametitle{READ}

  The converse of the \texttt{WRITE} statement is the \texttt{READ} statement.
  It follows the same rules and uses the same formatting syntax.

  \begin{lstlisting}[numbers=none]
read(*, '(I4, F8.2)') n, r
  \end{lstlisting}

  This command reads from standard input (keyboard) first a 4 digit integer, then a floating point number in the format xxxxx.xx

\end{frame}

\begin{frame}[fragile]
  \frametitle{Writing to CHARACTER}

  The first argument of the \texttt{WRITE} and \texttt{READ} statements is the 
  destination/source.
  The asterisk refers to standard input/output.
  But we can also direct it to something else.

  For example a character variable:

  \begin{lstlisting}
character(len=10) :: t
write(t, '(A5, I5.5)') "file_", 4
  \end{lstlisting}

  The above snippet would store the text "file\_00004" in the variable \texttt{t}.

\end{frame}

\begin{frame}[fragile]
  \frametitle{OPEN}

  To write to (or read from) a file, the file needs to be opened and closed.

  \begin{lstlisting}
integer, parameter :: my_unit = 300 ! Some number > 10
integer :: io_err

open(unit=my_unit, file="file.txt", action="WRITE", status="REPLACE", &
     iostat=io_err)
write(my_unit, '(A)') "This text is written to the file"
close(my_unit)
  \end{lstlisting}

  There are a lot more optional arguments to \texttt{OPEN}, google "Fortran OPEN" for a complete list.

\end{frame}

\begin{frame}[fragile]
  \frametitle{NAMELIST}

  A \texttt{NAMELIST} is a collection of variables that can be read from and
  written to a file very easily.

  First, the variables have to be declared and declared to be part of a namelist:

  \begin{lstlisting}
integer :: a
real :: r
real, dimension(10) :: t

namelist /NLIST/ a, r, t

open(unit=my_unit, file='file.txt', action='READ')
read(my_unit, nml=NLIST)
close(my_unit)
  \end{lstlisting}

\end{frame}

\begin{frame}[fragile]
  \frametitle{NAMELIST (cont'd)}

  In order to be read by the code on the previous page, the file "file.txt" must
  obey a certain format:

  \begin{lstlisting}
&NLIST
  a = 6,
  r = 4.51,
  t = 0.1, 0.2, 0.3, 0.4, 0.5, 0.6, 0.7, 0.8, 0.9, 1.0
/
  \end{lstlisting}

  A namelist like this can also be written by a \texttt{WRITE} statement.

\end{frame}

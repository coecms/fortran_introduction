
\begin{frame}[fragile]
  \frametitle{Procedures}

  Procedures are like little programs that the
  main program (or other procedures) can call on
  to preform small tasks.

  Splitting one huge project into many small procedures
  helps with code readability and reduces code duplication.

  Fortran knows two types of procedures

  \begin{enumerate}
    \item Subroutines
    \item Functions
  \end{enumerate}

\end{frame}

\begin{frame}[fragile]
  \frametitle{Subroutines}

  Subroutines look very similar to a program:

  \begin{lstlisting}
subroutine <routine name> (<arguments>)
    implicit none
    <declarations>
    <executable code>
end subroutine <routine name>
  \end{lstlisting}

  New are the \texttt{arguments}. 
  This is a list of variables that the caller has to pass to the
  subroutine in order for the subroutine to work correctly.

\end{frame}

\begin{frame}[fragile]
  \frametitle{Subroutine Example}

  \begin{lstlisting}
subroutine greet (cname)
    implicit none
    character(len=*) :: cname
    print *, "Hello " // trim(cname)
end subroutine greet
  \end{lstlisting}

  Here, the name of the subroutine is \texttt{greet}, and it takes one argument:
  \texttt{cname}, which is declared in line 3 as being a character with unknown length.

  This is one of the few times where Fortran doesn't need to know the length of the
  String. 
  \texttt{(len=*)} means that the routine should just take the string length as is.

\end{frame}

\begin{frame}[fragile]
  \frametitle{Calling a Subroutine}

  \begin{lstlisting}[numbers=none]
call greet("World")

call greet(cname = "Tony")

my_name = "Holger"
call greet(my_name)
  \end{lstlisting}

  These are three ways how to call the subroutine. 

\end{frame}

\begin{frame}[fragile]
  \frametitle{}

  \begin{lstlisting}
  \end{lstlisting}

\end{frame}

\begin{frame}[fragile]
  \frametitle{}

  \begin{lstlisting}
  \end{lstlisting}

\end{frame}

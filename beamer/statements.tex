\begin{frame}
  \frametitle{Basic Statements}
  In this section, I will explain
  \begin{itemize}
    \item How to assign values to variables
    \item Common sources of errors
    \item Loops
    \item Conditional Code
  \end{itemize}
\end{frame}

\begin{frame}[fragile]
  \frametitle{Basic Assignment}

  \begin{lstlisting}
  a = 3 + 4 * 2
  b = (3 + 4) * 2
  c = sqrt(2.0)
  i = i + 1
  \end{lstlisting}

  Single equals sign is assignment operator. 

  Right side is evaluated, assigned to variable on left.

  Line 4 is a common line that increments the value of \texttt{i} by one.

\end{frame}

\begin{frame}[fragile]
  \frametitle{Assignment Gotcha}

  \begin{lstlisting}
  REAL :: a

  a = 3 / 4
  print *, a  ! prints 0.0
  \end{lstlisting}

  3/4 is evaluated using integer division, ignoring the remainder.
  Then 0 is assigned to a, changed into a \texttt{REAL} type.

\end{frame}

\begin{frame}[fragile]
  \frametitle{Solution}

  Enforce the division to be a floating point division by turning
  at least one of numerator and denominator into \texttt{REAL}:

  \begin{lstlisting}
  a = 3.0 / 4
  a = float(3) / 4
  a = 3 / (4 * 1.0)
  \end{lstlisting}

  What doesn't work is converting the result of the division to \texttt{REAL}:

  \begin{lstlisting}[numbers=none]
  a = float(3 / 4)   ! Doesn't work
  \end{lstlisting}

\end{frame}

\begin{frame}[fragile]
  \frametitle{Loops}

  Loops repeat the same code several times, often have an iterator that changes
  its value every iteration.

  \begin{lstlisting}
  ! Old Style
  do 100 i = 1, 10
    print *, i
  100 continue

  ! New Style
  do i = 1, 10
    print *, i
  end do
  \end{lstlisting}
  These loops print the numbers from 1 to 10.

\end{frame}

\begin{frame}[fragile]
  \frametitle{Stride}

  Iterators don't have to increment by one every iteration.

  \begin{lstlisting}
  do i = 5, 20, 3   
  do i = 20, 10, -1
  \end{lstlisting}
  The third number is the stride, the amount by which the iterator
  changes in every iteration of the loop.

\end{frame}

\begin{frame}[fragile]
  \frametitle{DO WHILE}

  You can also repeat a loop while a condition is still valid:

  \begin{lstlisting}
  do while(j > 10)
    j = j - 2
  end do
  \end{lstlisting}

  Ensure that the loop will terminate eventually, or your program will hang.

\end{frame}

\begin{frame}[fragile]
  \frametitle{EXIT}

  The \texttt{exit} statement interrupts the loop and moves straight to the first
  statement after the loop:

  \begin{lstlisting}
  do i = 1, 10
    if (i > 5) exit
  end do
  \end{lstlisting}

\end{frame}

\begin{frame}[fragile]
  \frametitle{CYCLE}

  The \texttt{cycle} statement interrupts the current iteration of the loop
  and immediately starts with the next one.

  \begin{lstlisting}
  do i = 1, 100
    if (mod(i, 3) == 0) cycle ! Don't print multiples of 3
    print *, i
  end do
  \end{lstlisting}

\end{frame}

\begin{frame}[fragile]
  \frametitle{Named Loops}

  You can name loops.
  In this case, the \texttt{end do} statement has to repeat the name of the loop.

  \begin{lstlisting}
  print_loop : do i = 1, 10
    print *, i
  end do print_loop
  \end{lstlisting}

  This helps with readability when you have long and/or nested loops.

  This also helps you to clear which loop to \texttt{EXIT} or \texttt{CYCLE}.

\end{frame}

\begin{frame}[fragile]
  \frametitle{Nested Loops and Multi-Dimensional Arrays}

  Often nested loops are used to loop over all the indices of multi-dimensional arrays.

  \begin{lstlisting}
do k = 1, 100
    do j = 1, 100
        do i = 1, 100
            a(i, j, k) = 3.0 * b(i, j, k)
        end do
    end do
end do
  \end{lstlisting}

  For performance reasons, it is always best to have the \textbf{innermost} loop over the \textbf{first} index, and so on out.

  This has to do with the way Fortran stores multi-dimensional arrays.
\end{frame}


\begin{frame}[fragile]
  \frametitle{Conditional}

  \begin{lstlisting}[numbers=none]
  if (<condition>) <statement>
  \end{lstlisting}
  The \texttt{if} statement is used for conditional execution.
  Example:
  \begin{lstlisting}[numbers=none]
  if (i < 5) print *, "i is less than 5"
  \end{lstlisting}

  \texttt{<condition>} must evaluate to \texttt{.TRUE.} or \texttt{.FALSE.}, and only in the former case is the \texttt{<statement>} executed.

\end{frame}

\begin{frame}[fragile]
  \frametitle{IF Block}

  \begin{lstlisting}[numbers=none]
  if (<condition>) then
    <statement 1>
    <statement 2>
  end if
  \end{lstlisting}

  When the execution of several statements must be dependend on
  a condition, the keyword \texttt{THEN} is added after the condition.

  In that case, all statements until the \texttt{END IF} statement are
  executed.

\end{frame}

\begin{frame}[fragile]
  \frametitle{IF ELSE Block}

  \begin{lstlisting}[numbers=none]
  if (<condition>) then
    <true statement>
  else
    <false statement>
  end if
  \end{lstlisting}

  An \texttt{IF} block can be extended with an \texttt{ELSE} block that
  gets executed if and only if the initial condition evaluated to \texttt{.FALSE.}.

\end{frame}

\begin{frame}[fragile]
  \frametitle{ELSEIF Block}

  \begin{lstlisting}[numbers=none]
  if (<condition 1>) then
    <statement 1>
  elseif (<condition 2>) then
    <statement 2>
  else
    <statement 3>
  end if
  \end{lstlisting}

  If the initial condition is \texttt{.FALSE.} we can check for other conditions
  using the \texttt{ELSEIF} clause. 

  Note here that when \texttt{condition 1} is \texttt{.TRUE.}, only 
  \texttt{statement 1} is executed and \texttt{condition 2} never even checked.

  There is no limit on the number of \texttt{ELSEIF} clauses.

\end{frame}

\begin{frame}[fragile]
  \frametitle{Conditionals (cont'd)}

  At the end, \texttt{<condition>} has to be of type \texttt{LOGICAL}.

  But they can be combined (\texttt{A} and \texttt{B} are of type \texttt{LOGICAL}):

  \begin{table}
  \begin{tabular}{| c | l |}
    \hline
    \textbf{Code} & \textbf{Explanation} \cr
    \hline
    \texttt{.not. A} & inverts A \cr
    \texttt{A .or. B} & \texttt{.TRUE.} if at least one of \texttt{A} or \texttt{B} is. \cr
    \texttt{A .and. B} & \texttt{.TRUE.} if both \texttt{A} and \texttt{B} are. \cr
    \hline
  \end{tabular}
  \end{table}

\end{frame}

\begin{frame}[fragile]
  \frametitle{Conditionals (cont'd \#2)}

  Variables of all types can be compared to get \texttt{LOGICAL}:
  \begin{table}
  \begin{tabular}{| c | c | l |}
    \hline
    \textbf{F77 Code} & \textbf{F90 Code} & \textbf{\texttt{.TRUE.} if} \cr
    \hline
    \texttt{A .lt. B} & \texttt{A < B} & \texttt{A} is less than \texttt{B} \cr
    \texttt{A .gt. B} & \texttt{A > B} & \texttt{A} is greater than \texttt{B} \cr
    \texttt{A .le. B} & \texttt{A <= B} & \texttt{A} is less than or equal to \texttt{B} \cr
    \texttt{A .eq. B} & \texttt{A == B} & \texttt{A} is equal to \texttt{B} \cr
    \texttt{A .neq. B} & \texttt{A <> B} & \texttt{A} is not equal to \texttt{B} \cr
    \texttt{A .ge. B} & \texttt{A >= B} & \texttt{A} is greater than or equal to \texttt{B} \cr
    \hline
  \end{tabular}
  \end{table}

  \begin{lstlisting}
  if ( (i == 5) .and. (.not. (mod(j, 5) == 0))) then
    print *, "i is 5 and j is not a multiple of 5"
  end if
  \end{lstlisting}
\end{frame}

\begin{frame}[fragile]
  \frametitle{Floating Point Warning}
  
  Floating point numbers have limited precision. 
  This leads to rounding errors.
  In the familiar decimal system, we see this for example here:

  1 = 1/3 + 1/3 + 1/3 = 0.3333 + 0.3333 + 0.3333 = 0.9999

  Computers use binary, so their rounding errors might come as a surprise:

  \begin{lstlisting}[numbers=none]
  if (.not. ((0.1 + 0.2) == 0.3)) print *, "Rounding Error!"
  \end{lstlisting}

\end{frame}

\begin{frame}[fragile]
  \frametitle{CASE}

  Commonly, one wants to execute different statements depending on a single 
  variable.
  This could be done with lots of \texttt{ELSEIF} statements, but there is
  a more conveniant way:
  
  \begin{lstlisting}
  select case (var)
    case (-10)
      <what to do if var is equal to -10>
    case (10:)
      <what to do if var is more than or equal to 10>
    case default
      <what to do if none ot the other cases matched>
  end select
  \end{lstlisting}

\end{frame}


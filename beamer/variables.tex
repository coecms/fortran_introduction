\begin{frame}
  \frametitle{Variables}
  In this section, I will introduce
  \begin{itemize}
    \item The basic program structure
    \item The different variable types
    \item Kinds
    \item Custom Types
    \item Arrays
  \end{itemize}
\end{frame}

\begin{frame}[fragile]
  \frametitle{Basic Program Structure}
  \begin{lstlisting}[deletekeywords=name]
program <name>
  implicit none
  <variable declaration>
  <executable statements>
end program <name>
  \end{lstlisting}
\end{frame}

\begin{frame}
  \frametitle{Variable Types}
  \begin{table}
    \begin{tabular}{| l | l | l |}
    \hline
    \textbf{Fortran} & \textbf{Other} & \textbf{Examples} \cr
    \hline
    \texttt{LOGICAL} & bool(ean) & \texttt{.TRUE.} or \texttt{.FALSE.} \cr
    \texttt{INTEGER} & int & \texttt{0}, \texttt{10}, \texttt{-20} \cr
    \texttt{REAL} & float & \texttt{0.0}, \texttt{1.}, \texttt{.5}, \texttt{1.4e3} \cr
    \texttt{COMPLEX} &  & \texttt{(0.0, 1.0)} \cr
    \texttt{CHARACTER} & char, String & \texttt{"A"}, \texttt{"Hello World"} \cr
    \hline
    \end{tabular}
  \end{table}
\end{frame}

\begin{frame}[fragile]
  \frametitle{Parameters}
  Parameters, called constants in many other languages, are like variables, except
  their value cannot ever change.

  \begin{lstlisting}
REAL, PARAMETER :: pi = 3.141592
REAL :: e
PARAMETER(e = 2.71)
  \end{lstlisting}
\end{frame}

\begin{frame}[fragile]
  \frametitle{KIND}
  Variable KINDS are flavours of variables that change the precision and range of 
  values that can be stored in them.

  Usually, the KIND is the number of bytes required to store the data, though not all numbers are allowed.

  \begin{lstlisting}
REAL*8 :: var1
REAL(KIND=8) :: var2

INTEGER, PARAMETER :: dp = selected_real_kind(P = 15)
REAL(KIND=dp) :: var3
  \end{lstlisting}
\end{frame}

\begin{frame}[fragile]
  \frametitle{Character}

  Fortran can only understand fixed length Strings. 
  
  \begin{lstlisting}
CHARACTER*20 :: A
CHARACTER(LEN=10) :: B

A = "Hello World"   ! Actually stores "Hello World         "
B = A               ! Actually stores "Hello Worl"
  \end{lstlisting}

\end{frame}

\begin{frame}[fragile]
  \frametitle{Working with Characters}

  \begin{lstlisting}
print *, String1 // String2
print *, trim(String1) // " " // trim(String2)
print *, "Length of String1 is ", len(String1)
print *, "Length of String2 w/o trailing spaces is ", &
         len_trim(String1)
  \end{lstlisting}

\end{frame}

\begin{frame}[fragile]
  \frametitle{Types}

  You can create your own Variable types.

  \begin{lstlisting}
TYPE :: my_type
  INTEGER :: my_int
  REAL :: my_real
END TYPE my_type

TYPE(my_type) :: t

t % my_int = 5
t % my_real = 1.2
  \end{lstlisting}

\end{frame}

\begin{frame}[fragile]
  \frametitle{Arrays}
  Several values of the same type can be taken together in Arrays.

  \begin{lstlisting}
INTEGER :: a(10)
INTEGER, DIMENSION(10) :: b
INTEGER, DIMENSION(1:10) :: c
  \end{lstlisting}

  All three lines above do the same thing: They declare an array of 10 integers, with indices running from 1 to 10.

  To access certain values inside the array, you write the index in parentheses after the array name:

  \begin{lstlisting}[numbers=none]
print *, "The 5th element of a is ", a(5)
  \end{lstlisting}
\end{frame}

\begin{frame}[fragile]
  \frametitle{Arrays (continued)}
  \begin{lstlisting}
INTEGER, DIMENSION(-5:5) :: a
INTEGER, DIMENSION(10, 20) :: b
INTEGER, DIMENSION(:), ALLOCATABLE :: c
  \end{lstlisting}

  Line 1 shows how to create an array with indices that run from -5 to 5.

  Line 2 creates a two-dimensional array with 10 by 20 elements.

  Line 3 creates an allocatable array. In this case, it has one dimension, 
  but at compile time we don't yet know how big it needs to be.
  Before \texttt{c} can be used, it first has to be allocated, like so:

  \begin{lstlisting}[numbers=none]
  allocate(c(100))
  deallocate(c)
  \end{lstlisting}

\end{frame}
